\section{Related Work}  \label{sec:relevant-work}
Our research proposes and explores multi-tier data structures for managing the pending event set in sequential and optimistic parallel simulations. Specifically, we compare effectiveness of the data structures against our fine-tuned version of the Ladder Queue~\cite{tang-05} because it has shown to be very efficient for sequential DES. Recently, Franceschini et al~\cite{franceschini-15} compared several priority-queue based pending event data structures to evaluate their performance in the context of sequential DEVS simulations. They found that Ladder Queue outperformed every other priority queue based pending event data structure such as Sorted List, Minimal List, Binary Heap, Splay Tree, and Calendar Queue. Tang et al~\cite{tang-05} and Franceschini et al~\cite{franceschini-15} both use the classic Hold benchmark simulation model used in this study and described in Section ~\ref{sec:pholdModel}.

In contrast to earlier work, rather than using a linked list based implementation, we propose alternative implementation using dynamically growing arrays (i.e. std::vector). Furthermore, we trigger \textit{Bottom} to \textit{Ladder} re-bucketing only if the \textit{Bottom} has events at different timestamps to reduce inefficiencies. Our 2-tier Ladder Queue (\textbf{2tLadderQ}) is a novel enhancement to the Ladder Queue to enable its efficient use in optimistic parallel simulations.

Dickman et al~\cite{dickman-13} compare event list data structures that consisted of Splay Tree, STL Multiset and Ladder Queue. However, the focus of their paper was in developing a framework for handling pending event set data structure in shared memory PDES. A central component of their study was the identification of an appropriate data structure and design for the shared pending event set. Gupta et al~\cite{gupta-14} extended their implementation of Ladder Queue for shared memory Time Warp based simulation environment, so that it supports lock-free access to events in the shared pending event set. The modification involved the use of an unsorted lock-free queue in the underlying Ladder Queue structure. Marotta et al~\cite{marotta-16} have contributed to the study of pending event set data structures in threaded PDES through the design of the Non-Blocking Priority Queue (NBPQ) data structure. A pending event set data structure that is closely related to Calendar Queues with constant time performance.

In contrast to aforementioned efforts, our research focuses on distributed memory platforms in which each parallel process is single threaded. Consequently, our implementation does not involve thread synchronization issues. However, our 2-tier design has the ability to further reduce lock contention issues in multithreaded environments and could provide further performance boost. To the best of our knowledge, the Fibonnacci heap (\textbf{fibHeap}) and our 3-tier Heap (\textbf{3tHeap}) are unique data structures that have potential to be effective in simulations with high concurrency.