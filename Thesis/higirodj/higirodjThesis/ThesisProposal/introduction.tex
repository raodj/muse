\section{Introduction}  \label{sec:intro}
	Discrete event simulation (DES) is the simulation of a discrete event system that contains states that operate at discrete time steps~\cite{fishman-13}. At each point in time in a simulation, a virtual time-stamp is assigned to an event and the event precipitates a transition from one state to another state. This change in system state is used to represent the dynamic nature and behavior of a real-world system~\cite{fujimoto-90}. DES has been used in a variety of fields in academia, industry and the public sector as a tool to help inform our knowledge of discrete event systems and to improve decision-making processes~\cite{fishman-13}. DES provides an effective means for analyzing real or artificial systems without the constraint of limited resources such as time, financial costs, or safety. For example, the simulation of a battlefield environment can deliver insightful information to military planners on enemy troop movements, tactics and capabilities during strategic planning efforts~\cite{hill-01}. A discrete event simulation of the battlefield allows military leaders to examine the impacts of decisions without the real-world risks associated with committing forces to dangerous environments. 
    
	Parallelism in computing frameworks that support DES increase performance throughput that is needed to construct and execute large scale and complex simulation models. With the growth and prevalence of semiconductor technology, cheaper and powerful multi-processors can be instrumented to achieve greater computing power for parallel discrete event simulations (PDES). However, the speedup achieved using multi-core and multi-processor systems requires efficient parallel programs. 
    
    Sequential and parallel DES are designed as a set of logical processes (LPs) that interact with each other by exchanging and processing timestamped events or messages~\cite{jafer-13}. Events that are yet to be processed are called "pending events". Pending events must be processed by LPs in priority order to maintain causality, with event priorities being determined by their timestamps. Consequently, data structures for managing and prioritizing pending events play a critical role in ensuring efficient sequential and parallel simulations~\cite{jones-86,ronngren-97,brown-88,franceschini-15}. The effectiveness of data structures for event management is a conspicuous issue in larger simulations, where thousands or millions of events can be pending~\cite{carothers-2010,yeom-14}. Large pending event sets can arise when a model has many LPs or when each LP generates / processes many events. Overheads in managing pending events is magnified in fine grained simulations where the time taken to process an event is very short. Furthermore, the synchronization strategy used in PDES, Time Warp as described in section~\ref{sec:muse}, can further impact the effectiveness of the data structure due to additional processing required for rollback-based recovery.

\subsection{Motivations}
Many investigations have explored the effectiveness of a wide variety of data structures for managing the pending event set as discussed in Section~\ref{sec:relevant-work}. Among the various data structures, the Ladder Queue proposed by Tang et al.,~\cite{tang-05} has shown to be the most effective data structure for managing pending events~\cite{dickman-13,franceschini-15}, particularly in sequential DES. Accordingly, we aim to replace the heap-based data structures used in our Time Warp synchronized parallel simulator with the Ladder Queue. Section~\ref{sec:ladderQ} discusses our Ladder Queue implementation and its fine-tuning.
	The Ladder Queue outperformed our multi-tier heap-based data structures in certain sequential simulations, consistent with observations by other investigators~\cite{franceschini-15,yeom-14}. However, as detailed in section~\ref{sec:experiments}, the Ladder Queue was substantially slower in two cases -- \ding{182}
\underline{high concurrency}: larger number of concurrent events (i.e. events with same time-stamp) per LP, and \ding{183} Time Warp synchronized parallel simulations conducted on a distributed memory computing cluster. Conversely, our multi-tier data structures performed well in parallel simulations.

	To provide a good balance for both sequential and optimistic parallel simulations, we propose a significant change to the design of the Ladder Queue. Our revised data structure, discussed in Section~\ref{sec:2tLadderQ}, is called 2-tier Ladder Queue (2tLadderQ). Various configurations of the standard PHOLD benchmark are used to assess the effectiveness of the multi-tier data structures vs. our fine-tuned implementation of the Ladder Queue. The preliminary results from our experiments discussed in Section~\ref{sec:experiments} shows 2tLadderQ provides comparable performance in sequential simulations but outperforms the Ladder Queue in optimistic parallel simulations. Our 3-tier heap (3tHeap) outperforms our 2tLadderQ in high concurrency scenarios.
    

    
