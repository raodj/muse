\section{Introduction}
	Discrete event simulation (DES) is the simulation of a discrete event system that contains states that operate at discrete points in time~\cite{fishman-13}. At each point in time in a simulation, a virtual timestamp is assigned to an event and the event precipitates a transition from one state to another state. This change in system state is used to represent the dynamic nature and behavior of a real-world system~\cite{fujimoto-90}. DES has been used in a variety of fields in academia, industry and the public sector as a tool to help inform our knowledge of discrete event systems and to improve decision-making processes~\cite{fishman-13}. DES provides an effective means for analyzing real or artificial systems without the constraint of limited resources such as time, financial costs, or safety. For example, the simulation of a battlefield environment can deliver insightful information to military planners on enemy troop movements, tactics and capabilities during strategic planning efforts~\cite{hill-01}. A discrete event simulation of the battlefield allows military leaders to examine the impacts of decisions without the real-world risks associated with committing forces to dangerous environments. 
    
	Parallelism in computing frameworks that support DES increase performance throughput that is needed to construct and execute large scale and complex simulation models. With the growth and prevalence of semiconductor technology, cheaper and powerful multi-processors can be instrumented to achieve greater computing power for parallel discrete event simulations (PDES). However, the speedup achieved using multi-core and multi-processor systems requires efficient parallel programs. 
    
    Sequential and parallel DES are designed as a set of logical processes (LPs) that interact with each other by exchanging and processing timestamped events or messages~\cite{jafer-13}. Events that are yet to be processed are called "pending events". Pending events must be processed by LPs in priority order to maintain causality, with event priorities being determined by their timestamps. Consequently, data structures for managing and prioritizing pending events play a critical role in ensuring efficient sequential and parallel simulations~\cite{jones-86,ronngren-97,brown-88,franceschini-15}. The effectiveness of data structures for event management is a conspicuous issue in larger simulations, where thousands or millions of events can be pending~\cite{carothers-2010,yeom-14}. Large pending event sets can arise when a model has many LPs or when each LP generates / processes many events. Overheads in managing pending events is magnified in fine grained simulations where the time taken to process an event is very short. Furthermore, the synchronization strategy used in PDES, Time Warp as described in section 3, can further impact the effectiveness of the data structure due to additional processing required for rollback-based recovery.
    
