\section{Miami University Simulation Environment (MUSE)}
The implementation and assessment of the different data structures was conducted using our parallel simulation framework called MUSE. It was developed in C++ and uses the Message Passing Interface (MPI) library for parallel processing. MUSE uses Time Warp and standard state saving approach to accomplish optimistics synchronization of the LPs to maintain causality in event processing.

In a Time Warp based simulation such as MUSE, the simulation is organized as a set of LPs that interact with each other by exchanging virtual timestamped events. Each LP in a simulation maintains an input, output and state queue. The input queue is used to handle pending events that have yet to be processed. The output queue stores antimessages, which are events that are sent to other LPs to cancel out previously sent events or messages. The state queue stores the state of the LP at each discrete point in virtual simulation time. A Time Warp LP also maintains a local virtual time (LVT) tha.       
