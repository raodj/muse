\chapter{Conclusions}

Efficient data structures, \textit{i.e.,} priority queues for managing pending
event sets play a critical role in the overall performance of both
sequential and parallel simulations.  In the context of this study, we
broadly classified the queues into single-tiered (\textbf{heap} and
\textbf{ladderQ}) or multi-tiered (\textbf{2tHeap}, \textbf{fibHeap}, \textbf{3tHeap}, and
\textbf{2tLadderQ}) data structures based on their design.  Multi-tier data
structures organize pending events into tiers, with each tier possibly
implemented using different structures.  Organizing events into
multiple tiers decouples event management and Logical Process (LP)
scheduling permitting different algorithms and data structures to suit
the different needs~\cite{higiro2017multi}.

The comparative analysis used a fine-tuned version of the Ladder Queue
(\textbf{ladderQ}) proposed by Tang et al.~\cite{tang-05}.  Our objective in
fine-tuning was to reduce the runtime constants of the \textbf{ladderQ}
without significantly impacting its amortized O(1) time complexity. We realize the reduction in runtime constants by minimizing
memory management overheads -- \textit{i.e.,} \ding{182} favor few bulk
operations via \textbf{std::vector} than many small ones via \textbf{std::list}, and \ding{183} recycle memory or substructures rather than
reallocating them.  Using \textbf{std::vector} (\textit{i.e.,} dynamically growing
array) enables use of algorithms with lower time constants, such as:
\textbf{std::sort}, over \textbf{std::multiset} or binary heaps.  The bulk
memory operations do consume additional memory, but our analysis shows
that the performance gains significantly outweigh the extra memory
used.  Accordingly, other simulation kernels can significantly improve
overall performance by replacing linked-lists with dynamically growing
arrays~\cite{higiro2017multi}.

One challenge that arose during design of experiments was exploring
the large multidimensional parameter space in the \textbf{PHOLD} synthetic
benchmark and \textbf{PCS} simulation model.  Since large parameter spaces may arise with actual
simulation models.  We propose the use of Generalized Sensitivity
Analysis (GSA) to reduce the parameter space.  We propose the use of
Sobol random numbers to enable consistent exploration of the parameter
space.  GSA does require many simulations to be run to fully explore
the parameter space.  However, it was able to significantly narrow the
parameter space, \textit{i.e.,} from 9 down to 1.  GSA data shows that
concurrency per LP indicated by \textbf{eventsPerLP} parameter (\textit{i.e.,} batch
of events scheduled per LP), plays the most dominant role.  The data
was cross-verified using corellograms from longer simulations.  Similar
GSA analysis can be applied to other models and benchmarks enabling
consistent analysis for other aspects of simulations~\cite{higiro2017multi}.

The sequential and parallel simulation results showed that
\textbf{2tLadderQ} performs no worse than \textbf{ladderQ} in sequential
simulations (with \textsubscript{t2}\textit{k}=1).  Furthermore, the \textbf{2tLadderQ} performs
better in parallel simulations because of its design that enables
rapid cancellation of events during rollbacks.  In fact, the
\textbf{ladderQ} required aggressive throttling of optimism without which
it was impractical to use in scenarios with many cascading rollbacks.
These experiments were conducted with fine-grained settings (\textit{i.e.,}
\textbf{granularity}=0) and may vary with granularity.  However, GSA data
suggests that the variation with changing granularity would be small,
but may allowing relaxation of the time-window.  The results strongly
favor the general use of \textbf{2tLadderQ} over the \textbf{ladderQ}.
Furthermore, the multi-tier organization of \textbf{2tLadderQ} can further
reduce lock contention and consequent synchronization overheads in
multithreaded simulations~\cite{higiro2017multi}.

The experiments show that multi-tier queues also incur additional
overhead as part of the two step process.  Moreover, the runtime
constants play an important role -- for example, the Fibonacci heap
with its O(1) time complexity for many operations still did not
perform well in our benchmarks.  Consequently, in sequential
simulations, their advantages were realized in simulations that have
higher concurrency (\textit{i.e.,} larger batches of events) per LP.
The advantages of \textbf{3tHeap} is realized only when each
LP has 10 or more concurrent events at each time step. Such scenarios
with high \textbf{eventsPerLP} arises in epidemic models~\cite{yeom-14} and
detailed simulation models such as packet-level network
simulations~\cite{tang-05}. The simulations results using \textbf{PCS} showed that \textbf{3tHeap} is the desired scheduler queue for optimal runtime performance in sequential and optimistic parallel simulations.  However, further experimental analysis
with such models is necessary to validate effectiveness of the data
structures~\cite{higiro2017multi}.

The multi-tier data structures enjoy lower runtime constants for event
cancellation operations which play an influential role in Time Warp
synchronized parallel simulations.  Therefore, the multi-tier data
structures perform consistently better in optimistic parallel
simulations.  In overall summary, our analysis favor the use of \textbf{2tLadderQ} and \textbf{3tHeap}, as they are consistently effective in sequential and parallel simulations, with sequential results also bearing potential application to conservative and multithreaded simulations~\cite{higiro2017multi}.


\subsubsection{Future Work}

The results in this thesis provide us with an understanding of the effectiveness of multi-tier data structures for managing pending events in sequential \& optimistic parallel simulations. However, the thesis can be extended and our findings further strengthened through additional testing and analysis. An area worthy of further exploration is to determine whether or not our research findings are attributed to the characteristics and design of our multi-tier data structures or influenced by the framework that supported the testing. 

As such, we propose the implementation of our multi-tier data structure (i.e. \textbf{ladderQ}, \textbf{2tLadderQ} and \textbf{3tHeap} in a different programming language such as Java and comparing performance. Furthermore, the data structures should be assessed using different parallel simulation frameworks and experimental platforms. Franceschini et.al., evaluated different implementation of the pending event using their Ruby based discrete event simulator\cite{franceschini-15}. While, Luca Toscano et al., designed and implemented a Time Warp based parallel simulation framework developed in Erlang\cite{toscano2012parallel}. The approaches and frameworks presented in these studies could be applied to the evaluation of our multi-tier data structures. Lastly, we constrained our simulation models to the benchmark \textbf{PHOLD} and the 
\textbf{PCS} model. It would be useful to assess the performance of our data structures using a wider range of simulation models to examine the durability of our results

       






