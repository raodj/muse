\section{Simulation Models}  \label{sec:models}

\subsection{Parallel HOLD (PHOLD)} \label{sec:pholdModel}
The experimental analysis have been conducted using a parallelized version of the classic Hold synthetic benchmark called PHOLD. It has been used by many investigators because it has shown to effectively emulate the steady-state phase of a typical simulation~\cite{franceschini-15,tang-05}. Our PHOLD
implementation developed using MUSE provides several parameters
(specified as command-line arguments) summarized in table 1. The benchmark consists of a 2-dimensional grid of LPs specified via the \textbf{rows} and \textbf{cols} parameters. The LPs are evenly partitioned across the MPI-processes used for simulation. The \textbf{imbalance} parameter influences the partition, with larger values skewing the partition as shown in Figure\ref{fig:phold}(a). The \textbf{imbalance} parameter has no impact in sequential simulations.


\begin{table}[!ht]\centering
\textbf{\caption{Parameters in PHOLD benchmark}}\label{tab:phold-params}
\begin{tabular}{lp{2.2in}}
\toprule
Parameter & Description \\
\midrule

\textbf{rows} & Total number of rows in model. \\

\textbf{cols} & Total number of columns in model. \#LPs = \textbf{rows}

\texttimes\/ \textbf{cols} \\

\textbf{eventsPerLP} & Initial number of events per LP. \\

\textbf{delay} or $\lambda$ & Value used with distribution -- Lambda
($\lambda$) value for exponential distribution \textit{i.e.,}
$P(x|\lambda)=\lambda e^{-\lambda x}$. \\

\textbf{\%selfEvents} & Fraction of events LPs send to self \\

\textbf{granularity} & Additional compute load per event. \\

\textbf{imbalance} & Fractional imbalance in partition to have more LPs on a MPI-process. \\

\textbf{simEndTime} & GVT when simulation logically ends.\\

\bottomrule
\end{tabular}
\end{table}

\begin{figure*}[ht]
\begin{minipage}{0.32\linewidth}
\includegraphics[width=\linewidth]{images/skew_data}
\centerline{(a) Impact of \textbf{imbalance}}
\end{minipage} 
\begin{minipage}{0.32\linewidth}
\includegraphics[width=\linewidth]{images/gran_info}
\centerline{(b) Impact of \textbf{granularity}}
\end{minipage} 
\begin{minipage}{0.32\linewidth}
\includegraphics[width=\linewidth]{images/exp_delay_chart}
\centerline{(c) Impact of \textbf{delay} ($\lambda$)}
\end{minipage}
\textbf{\caption{Impact of varying key parameter values in the PHOLD model}}\label{fig:phold}
\end{figure*}
The PHOLD simulation commences with a fixed number of events for each LP, specified by the \textbf{eventsPerLP} parameter. For each event received by an LP a fixed number of trigonometric operations determined by \textbf{granularity} are performed to place CPU load. The impact of increasing the \textbf{granularity} parameter (no unit) is summarized in Figure 2(b) -- smaller values result in finer grained simulations. For each event, an LP schedules another event to a randomly chosen adjacent LP.  The \textbf{selfEvents} parameter controls the fraction of events that an LP schedules to itself.
The event timestamps are determined by a given \textbf{delay\--distrib} and \textbf{delay} or $\lambda$ parameters. Our experiments use an exponential distribution for timestamps, because it has shown to reflect event distribution commonly found in a broad range of simulation models~\cite{tang-05}. Time stamp of events is computed as $t_{recv}$ = LVT + 1 + $\lambda e^{-\lambda x}$. The impact of changing the $\lambda$ (\textit{i.e.,} \textbf{delay}) is shown in Figure \ref{fig:phold}(c) --smaller values of $\lambda$ provide a broader range of time stamp value for future events resulting in fewer concurrent events per LVT. Conversely, larger $\lambda$ values cause timestamps to be close to the current epoch, increasing both the number of concurrent events per LVT and the possibility of rollbacks. Section~\ref{sec:experiments} explores impact of these parameters on scheduler queue performance using 2,500 different configurations.

